% Systemdokumentation OOP3
% Unterlage für Studenten als Leitfaden für die Erstellung einer SystemDoku
% 17. Oktober 2022
% ---------------------------------------------------------------------------



% Dokumentklasse
% --------------
\documentclass[12pt,naustrian,a4widepaper]{scrartcl}   
% article style
%   - 11pt Schriftgroesse
%   - new austrian (neue Rechtschreibung)
%   - Papierformat A4
%   - pdf-hyperlinks


% Packages
% --------
\usepackage[utf8]{inputenc}  % fuer Umlaute, Ü
\usepackage[T1]{fontenc}
\usepackage{a4wide}
\usepackage{times}      % Times Schriften (zusammen mit fontencoding, s.o.)

\usepackage{babel}
\usepackage{graphicx}	  % für das Einbinden von Grafiken
\usepackage{color}      % für färbigen Text
\usepackage{framed}     % für (Text-) Rahmen
\usepackage{fancyhdr}   % für Kopf- und Fusszeilen
\usepackage{listings}   % für den Sourcecode
\usepackage{pdfpages}
\usepackage[margin=1.5cm]{geometry} % Verkleinert die Ränder

\pagestyle{fancy}       % Kopf- / Fusszeile aktivieren


% Beginn des Dokumentes
% ---------------------
\begin{document}

\subsection*{0) In 10 Sekunden: Compile-Fails zuerst}
\begin{itemize}
  \item Fehlende \texttt{;} / \texttt{)} / \texttt{\}} / falsche Klammern
  \item Tippfehler in Typen/Namen (z.B. \texttt{Tocket} vs \texttt{Ticket})
\end{itemize}

\subsection*{1) Polymorphie-Check (häufigster Punktverlust)}
\begin{itemize}
  \item \textbf{Virtueller Destruktor} in Basisklasse (wenn über Base-Pointer/Ref genutzt):
  \item \textbf{\texttt{override}} in abgeleiteten Klassen:
  \item \textbf{Slicing vermeiden:} keine Base-by-value Parameter/Returns; stattdessen
  \texttt{Base\&}, \texttt{Base*}, \texttt{std::unique\_ptr<Base>}.
\end{itemize}

\subsection*{2) Copy/Assign korrekt (Rule of 0/5)}
\begin{itemize}
  \item Wenn Copy/Assign \textbf{verboten} sein soll:
\begin{verbatim}
Base(Base const&) = delete;
Base& operator=(Base const&) = delete;
\end{verbatim}
  \item Bei Ownership: \textbf{\texttt{std::unique\_ptr}} bevorzugen (statt roher Pointer).
\end{itemize}

\subsection*{3) Ownership / Speicher (Factories \& Container)}
\begin{itemize}
  \item Factory-Rückgabe (typisch):
\begin{verbatim}
std::unique_ptr<Base> Create(...);
\end{verbatim}
  \item Container-Speicherung:
\begin{verbatim}
std::vector<std::unique_ptr<Base>> v;
\end{verbatim}
  \item Nie: \texttt{return \&local;} oder \texttt{new} ohne klaren Besitzer.
  \item Base-Pointer korrekt zuweisen:
\begin{verbatim}
Base* p = &derived;   // nicht: Base* p = derived;
\end{verbatim}
\end{itemize}

\subsection*{4) Sichtbarkeit \& \texttt{const}}
\begin{itemize}
  \item Interface-Methoden: \texttt{public}
  \item Member: fast immer \texttt{private}
\end{itemize}

\subsection*{Pattern-spezifische Mini-Checklisten}

\subsubsection*{Template Method}
\begin{itemize}
  \item Template-Funktion definiert Ablauf, ruft Hooks:
\begin{verbatim}
void Print() const { header(); DoPrint(); footer(); }
protected: virtual void DoPrint() const = 0;
\end{verbatim}
  \item Hook ist \texttt{virtual}, häufig \texttt{= 0}.
\end{itemize}

\subsubsection*{Factory Method}
\begin{itemize}
  \item Creator-Interface:
\begin{verbatim}
virtual std::unique_ptr<Product> Create(...) = 0;
\end{verbatim}
  \item Konkrete Creator erstellen konkrete Products.
  \item High-Level nutzt \textbf{Creator-Abstraktion} (nicht \texttt{new ConcreteProduct} im High-Level).
\end{itemize}

\subsubsection*{Strategy}
\begin{itemize}
  \item Strategy-Interface + austauschbar:
\begin{verbatim}
struct Strategy { virtual ~Strategy()=default; virtual void Run()=0; };
\end{verbatim}
  \item Kontext hält \texttt{unique\_ptr<Strategy>} oder Referenz.
\end{itemize}

\subsubsection*{Observer}
\begin{itemize}
  \item \texttt{Attach / Detach / Notify}
  \item Bei \texttt{Notify}: Iterator-Invalidierung vermeiden (z.B. Liste kopieren)
  \item Ownership-Zyklen vermeiden (oft \texttt{weak\_ptr} bei bidirektional)
\end{itemize}

\subsubsection*{Visitor}
\begin{itemize}
  \item \texttt{Accept(Visitor\&)} in Elementen
  \item \texttt{Visit(Concrete\&)} im Visitor
  \item Kein \texttt{dynamic\_cast} nötig (Doppeldispatch korrekt)
\end{itemize}

\subsubsection*{Decorator}
\begin{itemize}
  \item Decorator erbt von Component und hält Component:
\begin{verbatim}
std::unique_ptr<Component> inner_;
\end{verbatim}
  \item Delegieren + erweitern (Aufruf weiterleiten nicht vergessen)
\end{itemize}

\subsection*{5) DIP-Schnellcheck (falls gefordert)}
\begin{itemize}
  \item High-Level soll nur \textbf{Abstraktionen} kennen, keine konkreten Klassen.
  \item Lösung: \textbf{Dependency Injection} (Ctor/Setter/Parameter) oder Factory.
\end{itemize}

\subsection*{30-Sekunden Final Scan (immer machen)}
\begin{itemize}
  \item Base hat \texttt{virtual \textasciitilde Base()}?
  \item \texttt{override} überall korrekt?
  \item Pointer-Zuweisung mit \texttt{\&}?
  \item \texttt{operator=} liefert \texttt{T\&}?
  \item Ownership klar (\texttt{unique\_ptr})?
  \item Keine Base-by-value Übergaben/Returns?
\end{itemize}


\end{document}
