% Systemdokumentation OOP3
% Unterlage für Studenten als Leitfaden für die Erstellung einer SystemDoku
% 17. Oktober 2022
% ---------------------------------------------------------------------------



% Dokumentklasse
% --------------
\documentclass[12pt,naustrian,a4widepaper]{scrartcl}   
% article style
%   - 11pt Schriftgroesse
%   - new austrian (neue Rechtschreibung)
%   - Papierformat A4
%   - pdf-hyperlinks


% Packages
% --------
\usepackage[utf8]{inputenc}  % fuer Umlaute, ß
\usepackage[T1]{fontenc}
\usepackage{a4wide}
\usepackage{times}      % Times Schriften (zusammen mit fontencoding, s.o.)

\usepackage{babel}
\usepackage{graphicx}	  % für das Einbinden von Grafiken
\usepackage{color}      % für färbigen Text
\usepackage{framed}     % für (Text-) Rahmen
\usepackage{fancyhdr}   % für Kopf- und Fusszeilen
\usepackage{listings}   % für den Sourcecode



\pagestyle{fancy}       % Kopf- / Fusszeile aktivieren


\lstset{
	language={C++},
	basicstyle=\footnotesize\ttfamily,
	keywordstyle=\color{blue},%\bfseries,
	commentstyle=\color{green},
	frame=single,
	linewidth=16cm,
	breaklines=false,
	tabsize=3,
	numbers=left, numberstyle=\tiny, stepnumber=1, numbersep=5pt
}

% Seitenspiegel
% -------------

\typearea{8}	% Festlegung des Seitenspiegels gem. Koma. 4..groß, 9..klein


% Kopfzeile
% ---------
\lhead{{\footnotesize{s. Offenberger, S. Vogelhuber}}}   %  (links)
\chead{{\footnotesize{Systemdokumentation - Fuhrpark}}} %  (mitte)
\rhead{{\footnotesize{Seite \thepage}}}      %  (rechts)

% Fusszeile
% ---------
\lfoot{}  % links
\cfoot{}  % mitte 
\rfoot{}  % rechts

% Absatzformatierung
% ------------------
\setlength{\parindent}{0cm}   % Einrückung der 1. Zeile jedes Absatzes
\setlength{\parskip}{10pt}    % Abstand zwischen den Absätzen


% Package für Hyperlinks (mit pdf-Optionen)
% -----------------------------------------
\usepackage[
urlcolor=blue,		% blaue weblinks
linkcolor=black,	% interne Links sind schwarz
colorlinks=true,        % links werden eingefärbt
pdfstartview=FitH,      % PDF-Anzeige: Fensterbreite
pdfborder={0 0 0},      % keine Umrandung um links
pdftitle   ={Systemdokumentation},% Referenzen in der Pdf-Datei
pdfauthor  ={M. Mustermann, S. Sorglos},
pdfsubject ={Systemdoku},
pdfcreator ={Der Creator},
pdfproducer={Der Producer},
pdfkeywords={Dokumentation, Systemdokumentation}
]{hyperref}


% Beginn des Dokumentes
% ---------------------
\begin{document}

\selectlanguage{naustrian}   % oder "american" für engl. Texte


% Titelblatt
% ----------
\title {\vspace{1cm}
       \includegraphics[width=8cm]{./Images/FhOOeLogoOkt2009_HSD_Rot_pastell}\\
       \vspace{2cm}
       {\textbf{Systemdokumentation\\Projekt Fuhrpark}}\\
       \vspace{5mm}
       {\small{Version 1.0}}\\
       \vspace{5mm}
}

\author{\small{S. Offenberger, S. Vogelhuber}}
\date  {\small{Hagenberg, \today}}
\maketitle

%\begin{abstract}
%Dieses Dokument zeigt den prinzipiellen Aufbau einer Systemdokumentation für Software-Projekte. Die einzelnen Kapitel sind mit Kommentaren versehen, welche die Struktur und den Inhalt erläutern. 
%\end{abstract}

\clearpage

% Inhalts-, Tabellen- und Bildverzeichnis (werden generiert)
% ----------------------------------------------------------
\tableofcontents
% \listoftables
% \listoffigures
\clearpage



\section{Organisatorisches}

\subsection{Team}
\begin{itemize}
	\item Simon Offenberger, Matr.-Nr.: S2410306027, E-Mail: Simon.Offenberger@fh-hagenberg.at	
	\item Susi Sorglos, Matr.-Nr.: yyyy, E-Mail: Susi.Sorglos@fh-hagenberg.at	
\end{itemize}

\subsection{Aufteilung der Verantwortlichkeitsbereiche}
\begin{itemize}
	\item Simon Offenberger
		\begin{itemize}
			\item Design Klassendiagramm
			\item Implementierung und Test der Klassen: 
			\begin{itemize}
				\item Object,
				\item RecordEntry, 
				\item DriveRecord,
				\item Vehicle,
			\end{itemize}
			\item Implementierung des Testtreibers
			\item Dokumentation
		\end{itemize}
	\item Simon Vogelhuber
		\begin{itemize}
			\item Design Klassendiagramm
			\item Implementierung und Komponententest der Klassen: 
			\begin{itemize}
				\item Garage
				\item Car, 
				\item Bike und  
				\item Truck 
			\end{itemize}
			\item Implementierung des Testtreibers
			\item Dokumentation
		\end{itemize}	
\end{itemize}

\subsection{Aufwand}
	
	\begin{itemize}
		\item Simon Offenberger: geschätzt 10 Ph / tatsächlich x Ph
		\item Simon Vogelhuber:  geschätzt x Ph / tatsächlich x Ph
	\end{itemize}

\clearpage
\section{Anforderungsdefinition (Systemspezifikation)}

In diesem System werden die Tiere eines Zoo's abgebildet und dort gespeichert. Die Tiere speichern das Gewicht in Kilogramm als Ganzzahl und werden mit einer forlaufenden Nummer identifiziert. Sie besitzen eine gemeinsame Schnittstelle die folgende Funktionalität liefert:
\begin{itemize}
	\item Gib einen Laut (entsprechende Ausgabe auf std::cout).
	\item Liefere einen String mit den gespeicherten Attributen.
	\item Erstelle einen Klon von sich selbst.
\end{itemize}

Der Zoo speichert alle Tiere und besitzt die Tier-Objekte nach dem Hinzufügen. Via Zugriffsmethoden kann auf die Tiere zugegriffen werden und eine String-Methode liefert eine Repräsentation aller Tiere im Zoo mit allen Attributen als eine abgeschlossene Zeichenkette. Zusätzlich kann der Zoo inklusive all seiner Tiere kopiert und zugewiesen werden.

\clearpage
\section{Systementwurf}

\subsection{Klassendiagramm}
\color{blue}
Hier wird das Klassendiagramm eingefügt. Sollte dieses nicht auf eine A4-Seite passen, so kann es in eine eingene pdf-Datei ausgelagert werden.
Verweisen Sie an dieser Stelle auf diese Datei.
\color{black}


\subsection{Designentscheidungen}
	Im Klassendiagramm wurde der Polymorphismus angewendet, um unterschiedliche Tierarten mit der gemeinsamen Schnittstelle 'Animal' anzusprechen. Die Klasse 'Zoo' speichert einen Container mit der abstrakte Basisklasse 'Animal' als Elementtyp und kann somit alle bestehenden und auch neuen Tierarten verwalten, die sich von der gemeinsamen Basisklasse 'Animal' ableiten.

\color{blue}
Designentscheidungen sind von entscheidender Bedeutung für die Qualität und den Erfolg einer Softwareanwendung. Sie beeinflussen nicht nur die technische Umsetzung, sondern auch die Fähigkeit der Anwendung, zukünftigen Anforderungen gerecht zu werden und Änderungen effizient zu bewältigen.\\
Sie beantworten meist folgende Fragen:\\
\begin{itemize}
	\item Warum wurde die Klassenhierarchie so gewählt?
	\item Wurden Design Pattern verwendet und warum?
	\item Wurde Abstraktion und der Polymorphismus angewendet?
	\item Wie kann die Klassenstruktur einfach erweitert werden?
	\item 
\end{itemize}

\color{black}

\section{Dokumentation der Komponenten (Klassen)}
\color{blue}
Die Dokumentation der einzelnen Klassen und Komponenten erfolgt direkt im Quellcode mit Doxygen-Kommentaren. Erzeugen Sie danach 
eine HTML-Doku und verweisen Sie auf die Start-HTML-Datei.\\

\color{black}
Die HTML-Startdatei befindet sich im Verzeichnis \href{run:./../doxy/html/index.html}{./../doxy/html/index.html}

\clearpage
\section{Testprotokollierung}

\begin{verbatim}
Visual Leak Detector read settings from: C:\Program Files (x86)\Visual Leak Detector\vld.ini
Visual Leak Detector Version 2.5.1 installed.
Testing Invalid Animal Creation...
[PASS] Exception caught: Weight must be a positive number
[FAIL] Expected exception for invalid cat weight not thrown.
Testing Null Animal Addition to Zoo...
[PASS] Exception caught: null_pointer param in Zoo::Add(...)
Testing Animal Weights...
[PASS] Animal weight test passed.
[PASS] Animal weight test passed.
[PASS] Animal weight test passed.
Testing Animal IDs...
[PASS] Animal ID test passed.
[PASS] Animal ID test passed.
[PASS] Animal ID test passed.
Testing Animal Sounds...
[PASS] Animal sound test passed.
[PASS] Animal sound test passed.
[PASS] Animal sound test passed.
Testing Animal ToString...
[PASS] Animal ToString test passed.
[PASS] Animal ToString test passed.
[PASS] Animal ToString test passed.
Testing Animal Cloning...
[PASS] Animal cloning test passed.
[PASS] Animal cloning test passed.
Testing Zoo Contents...
[PASS] Zoo ToString test passed.
No memory leaks detected.
Visual Leak Detector is now exiting.
\end{verbatim}

\clearpage
\section{Quellcode}
\color{blue}
Die Klassen werden entsprechend der Klassenhierarchie von oben nach unten angegeben. Zuerst die Header-Datei, gefolgt von der Implementierung.
Der Testtreiber (oder Client-Klassen) werden am Ende angegeben.
\color{black}

% Literaturverzeichnis
% --------------------
%\begin{thebibliography}{99}
%\bibitem{Pomberger} Pomberger G., Blaschek G. : \textit{Software Engineering: Prototyping und objektorientert Software-Entwicklung}. Hanser, 1996
%
%\end{thebibliography}


% Ende des Dokuments
% ------------------
\end{document}
