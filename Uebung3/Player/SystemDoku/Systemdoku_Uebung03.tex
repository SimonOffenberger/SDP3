% Systemdokumentation OOP3
% Unterlage für Studenten als Leitfaden für die Erstellung einer SystemDoku
% 17. Oktober 2022
% ---------------------------------------------------------------------------



% Dokumentklasse
% --------------
\documentclass[12pt,naustrian,a4widepaper]{scrartcl}   
% article style
%   - 11pt Schriftgroesse
%   - new austrian (neue Rechtschreibung)
%   - Papierformat A4
%   - pdf-hyperlinks


% Packages
% --------
\usepackage[utf8]{inputenc}  % fuer Umlaute, Ü
\usepackage[T1]{fontenc}
\usepackage{a4wide}
\usepackage{times}      % Times Schriften (zusammen mit fontencoding, s.o.)

\usepackage{babel}
\usepackage{graphicx}	  % für das Einbinden von Grafiken
\usepackage{color}      % für färbigen Text
\usepackage{framed}     % für (Text-) Rahmen
\usepackage{fancyhdr}   % für Kopf- und Fusszeilen
\usepackage{listings}   % für den Sourcecode
\usepackage{pdfpages}

\pagestyle{fancy}       % Kopf- / Fusszeile aktivieren

\definecolor{failred}{rgb}{0.7, 0.0, 0.0}
\definecolor{okgreen}{rgb}{0.0, 0.5, 0.0}
\definecolor{gray}{rgb}{0.0, 0.5, 0.0}

\lstdefinelanguage{TestOutput}{
    morekeywords={},
    morecomment=[l]{//},
    morestring=[b]",
    sensitive=false,
}

\lstdefinestyle{teststyle}{
    language=TestOutput,
    basicstyle=\ttfamily\footnotesize,
    keywordstyle=\color{black},
    commentstyle=\color{gray},
    stringstyle=\color{black},
    showstringspaces=false,
    breaklines=true,
    frame=single,
    numbers=left,
    numberstyle=\tiny\color{gray},
    postbreak=\mbox{\textcolor{red}{$\hookrightarrow$}\space},
    literate={OK}{{\textcolor{okgreen}{OK}}}2
             {Fail}{{\textcolor{failred}{Fail}}}4,
}

\lstdefinestyle{cppstyle}{
  language=C++,
  basicstyle=\ttfamily\tiny,
  keywordstyle=\color{blue},
  commentstyle=\textcolor{okgreen},
  stringstyle=\color{red},
  numbers=left,
  numberstyle=\tiny\color{gray},
  stepnumber=1,
  breaklines=true,
  frame=single
}



% Seitenspiegel
% -------------

\typearea{10}	% Festlegung des Seitenspiegels gem. Koma. 4..groß, 9..klein


% Kopfzeile
% ---------
\lhead{{\footnotesize{s. Offenberger, S. Vogelhuber}}}   %  (links)
\chead{{\footnotesize{Systemdokumentation - Music/VideoPlayer Adapter}}} %  (mitte)
\rhead{{\footnotesize{Seite \thepage}}}      %  (rechts)

% Fusszeile
% ---------
\lfoot{}  % links
\cfoot{}  % mitte 
\rfoot{}  % rechts

% Absatzformatierung
% ------------------
\setlength{\parindent}{0cm}   % Einrückung der 1. Zeile jedes Absatzes
\setlength{\parskip}{10pt}    % Abstand zwischen den Absätzen


% Package für Hyperlinks (mit pdf-Optionen)
% -----------------------------------------
\usepackage[
urlcolor=blue,		% blaue weblinks
linkcolor=black,	% interne Links sind schwarz
colorlinks=true,        % links werden eingefärbt
pdfstartview=FitH,      % PDF-Anzeige: Fensterbreite
pdfborder={0 0 0},      % keine Umrandung um links
pdftitle   ={Systemdokumentation},% Referenzen in der Pdf-Datei
pdfauthor  ={M. Mustermann, S. Sorglos},
pdfsubject ={Systemdoku},
pdfcreator ={Der Creator},
pdfproducer={Der Producer},
pdfkeywords={Dokumentation, Systemdokumentation}
]{hyperref}


% Beginn des Dokumentes
% ---------------------
\begin{document}

\selectlanguage{naustrian}   % oder "american" für engl. Texte

\includepdf[pages = 1-3]{Deckblatt_Uebung03.pdf}


% Titelblatt
% ----------
\title {\vspace{1cm}
       \includegraphics[width=8cm]{./Images/FhOOeLogoOkt2009_HSD_Rot_pastell}\\
       \vspace{2cm}
       {\textbf{Systemdokumentation\\Projekt Music/VideoPlayer Adapter}}\\
       \vspace{5mm}
       {\small{Version 1.0}}\\
       \vspace{5mm}
}

\author{\small{S. Offenberger, S. Vogelhuber}}
\date  {\small{Hagenberg, \today}}
\maketitle

%\begin{abstract}
%Dieses Dokument zeigt den prinzipiellen Aufbau einer Systemdokumentation für Software-Projekte. Die einzelnen Kapitel sind mit Kommentaren versehen, welche die Struktur und den Inhalt erläutern. 
%\end{abstract}

\clearpage

% Inhalts-, Tabellen- und Bildverzeichnis (werden generiert)
% ----------------------------------------------------------
\tableofcontents
% \listoftables
% \listoffigures
\clearpage



\section{Organisatorisches}

\subsection{Team}
\begin{itemize}
	\item Simon Offenberger, Matr.-Nr.: S2410306027, E-Mail: S2410306027@fhooe.at
	\item Simon Vogelhuber, Matr.-Nr.: S2410306014, E-Mail: s2410306014@fhooe.at	
\end{itemize}

\subsection{Aufteilung der Verantwortlichkeitsbereiche}
\begin{itemize}
	\item Simon Offenberger
		\begin{itemize}
			\item Design Klassendiagramm
			\item Implementierung und Test der Klassen: 
			\begin{itemize}
				\item Client,
				\item VideoPlayerAdapter, 
				\item VideoPlayer,
				\item Video,
				\item EVideoFormat,
			\end{itemize}
			\item Implementierung des Testtreibers
			\item Dokumentation
		\end{itemize}
	\item Simon Vogelhuber
		\begin{itemize}
			\item Design Klassendiagramm
			\item Implementierung und Komponententest der Klassen: 
			\begin{itemize}
				\item IPlayer
				\item MusicPlayerAdapter, 
				\item MusicPlayer, 
				\item Song 
			\end{itemize}
			\item Implementierung des Testtreibers
			\item Dokumentation
		\end{itemize}	
\end{itemize}

\subsection{Aufwand}
	
	\begin{itemize}
		\item Simon Offenberger: geschätzt 12 Ph / tatsächlich 11 Ph
		\item Simon Vogelhuber:  geschätzt 9 Ph / tatsächlich 9 Ph
	\end{itemize}

\clearpage
\section{Anforderungsdefinition (Systemspezifikation)}
Für die Implementierung wurden die Header von MusicPlayer, VideoPlayer und IPlayer Interface vorgegeben.
Die Anforderung bestand darin einen Client eine gemeinsame Schnittstelle zum Ansprechen von
MusicPlayer sowie VideoPlayer zu bieten.  
Die Schnittstelle soll folgende Funktionen bereitstellen.
\subsection{IPlayer Interface Anforderung}
\begin{itemize}
	\item Play
	\begin{itemize}
		\item Spielt das Video bzw. den Song des entsprechenden Players -> Ausgabe auf COUT
	\end{itemize}
	\item VolInc
	\begin{itemize}
		\item Diese Methode soll die Lautstärke des Players um 1 erhöhen.
	\end{itemize}
	\item VolDec
	\begin{itemize}
		\item Diese Methode soll die Lautstärke des Players um 1 verringern.
	\end{itemize}
	\item Stop
	\begin{itemize}
		\item Stoppt die Wiedergabe
	\end{itemize}
	\item Next
	\begin{itemize}
		\item Wechselt den aktuellen Titel auf den nächsten in der Liste
	\end{itemize}
	\item Next
	\begin{itemize}
		\item Wechselt den aktuellen Titel auf den vorherigen in der Liste
	\end{itemize}
	\item Select
	\begin{itemize}
		\item Wählt einen Titel über den Namen aus
	\end{itemize}
\end{itemize}
\clearpage
\subsection{VideoPlayer Anforderung}
Folgende Anforderungen müssen die Methoden des VideoPlayers bereitstellen:
\begin{itemize}
	\item Play
	\begin{itemize}
		\item Spielt das Video ab -> Ausgabe auf COUT
	\end{itemize}
	\item Stop
	\begin{itemize}
		\item Stopt das Video -> Ausgabe auf COUT
	\end{itemize}
	\item First
	\begin{itemize}
		\item Wechsel auf den ersten Titel in der Playlist
		\item gibt true zurück wenn dies erfogreich ist
		\item gibt false wenn kein Titel in der Playlist ist
	\end{itemize}
	\item Next
	\begin{itemize}
		\item Wechsel auf den nächsten Titel in der Playlist
		\item gibt true zurück wenn dies erfogreich ist
		\item gibt false wenn kein weiterer Titel in der Playlist ist
	\end{itemize}
	\item CurIndex
	\begin{itemize}
		\item Liefert den aktuellen Index der Playlist
	\end{itemize}
	\item CurVideo
	\begin{itemize}
		\item Liefert den aktuellen Title als string
	\end{itemize}
	\item SetVolume
	\begin{itemize}
		\item Setzt die Lautstärke des Titles max 50 min 0
	\end{itemize}
	\item GetVolume
	\begin{itemize}
		\item Liefert die aktuelle Lautstärke
	\end{itemize}
	\item Add
	\begin{itemize}
		\item Fügt und erzeugt ein Video an die Playlist hinten an
	\end{itemize}
\end{itemize}
\clearpage
\subsection{VideoPlayer Anforderung}
Folgende Anforderungen müssen die Methoden des MusicPlayers bereitstellen:
\begin{itemize}
	\item Start
	\begin{itemize}
		\item Spielt den Song ab -> Ausgabe auf COUT
	\end{itemize}
	\item Stop
	\begin{itemize}
		\item Stopt den Song -> Ausgabe auf COUT
	\end{itemize}
	\item SwitchNext
	\begin{itemize}
		\item Wechsel auf den nächsten Titel in der Playlist am Ende wird mit den ersten fortgesetzt
	\end{itemize}
	\item GetCurIndex
	\begin{itemize}
		\item Liefert den aktuellen Index der Playlist
	\end{itemize}
	\item Find
	\begin{itemize}
		\item Sucht nach einem Titel und wählt ihn aus
		\item gibt true wenn Titel gefunden wurde
		\item gibt false wenn Titel nicht gefunden wurde
	\end{itemize}
	\item GetCount
	\begin{itemize}
		\item Gibt die Anzahl der Lieder in der Playlist zurück
	\end{itemize}
	\item IncreaseVol
	\begin{itemize}
		\item erhöht die Lautstärke um einen bestimmten Wert (max 100)
	\end{itemize}
	\item DecreaseVol
	\begin{itemize}
		\item reduziert die Lautstärke um einen bestimmten Wert (min 0)
	\end{itemize}
	\item Add
	\begin{itemize}
		\item Fügt und erzeugt ein Video an die Playlist hinten an
	\end{itemize}
\end{itemize}
\section{Systementwurf}
\subsection{Klassendiagramm}
\includegraphics[height=22cm]{./Player.jpg}
\newpage

\subsection{Designentscheidungen}
Die Klassen Video und Song wurden so umgesetzt, dass diese für die Speicherung der spezifischen Daten eingesetzt werden.
Hier wird in den Playerklassen ein Container von Videos bzw. Songs gespeichert.
Für die Bereitstellung eines gemeinsamen Interfaces für den Client wurden Adapter für den Music- bzw. Video Player implementiert.
Dieser Adapter speichern intern nur eine Referenz auf den tatsächlichen Players. 
Dies ermöglicht es den Player selbst als auch den Adapter simultan zu verwenden.
Im Adapter müssten die Funktion der Player so angewandt und kombiniert werden, dass für beide Player über das Interface diesselbe 
Funktionalität zur Verfügung steht.

Die Gemeinsamen funktionen des Interfaces wurde im Client getestet.
Alle anderen Klassen wurden im main getestet.

In der Übung wurde nachgefragt ob die starre Ausgabe auf cout, über einen Parameter in der Methode
ausgetauscht werden kann, aber nach Absprache mit Herrn Wiesinger dürfen keine Veränderungen vorgenommen werden.
Somit musste im Testtreiber cout umgeleitet werden um einen sinnvollen Testtreiber zu schreiben.


\color{black}

\section{Dokumentation der Komponenten (Klassen)}
Die HTML-Startdatei befindet sich im Verzeichnis \href{run:./../doxy/html/index.html}{./../doxy/html/index.html}

\clearpage
\section{Testprotokollierung}
\lstinputlisting[style=teststyle]{../TestOutput.txt}

\clearpage
\section{Quellcode}

\subsection{Object.hpp}
\lstinputlisting[style = cppstyle]{../Object.hpp}
\clearpage
\subsection{Client.hpp}
\lstinputlisting[style = cppstyle]{./../Client.hpp}
\clearpage

\subsection{Client.cpp}
\lstinputlisting[style = cppstyle]{./../Client.cpp}
\clearpage

\subsection{IPlayer.hpp}
\lstinputlisting[style = cppstyle]{./../IPlayer.hpp}
\clearpage

\subsection{MusicPlayerAdapter.hpp}
\lstinputlisting[style = cppstyle]{./../MusicPlayerAdapter.hpp}
\clearpage

\subsection{MusicPlayerAdapter.cpp}
\lstinputlisting[style = cppstyle]{./../MusicPlayerAdapter.cpp}
\clearpage

\subsection{MusicPlayer.hpp}
\lstinputlisting[style = cppstyle]{./../MusicPlayer.hpp}
\clearpage

\subsection{MusicPlayer.cpp}
\lstinputlisting[style = cppstyle]{./../MusicPlayer.cpp}
\clearpage

\subsection{Song.hpp}
\lstinputlisting[style = cppstyle]{./../Song.hpp}
\clearpage

\subsection{Song.cpp}
\lstinputlisting[style = cppstyle]{./../Song.cpp}
\clearpage

\subsection{VideoPlayerAdapter.hpp}
\lstinputlisting[style = cppstyle]{./../VideoPlayerAdapter.hpp}
\clearpage

\subsection{VideoPlayerAdapter.cpp}
\lstinputlisting[style = cppstyle]{./../VideoPlayerAdapter.cpp}
\clearpage

\subsection{VideoPlayer.hpp}
\lstinputlisting[style = cppstyle]{./../VideoPlayer.hpp}
\clearpage

\subsection{VideoPlayer.cpp}
\lstinputlisting[style = cppstyle]{./../VideoPlayer.cpp}
\clearpage

\subsection{Video.hpp}
\lstinputlisting[style = cppstyle]{./../Video.hpp}
\clearpage

\subsection{Video.cpp}
\lstinputlisting[style = cppstyle]{./../Video.cpp}
\clearpage

\subsection{EVideoFormat.hpp}
\lstinputlisting[style = cppstyle]{./../EVideoFormat.hpp}
\clearpage
\subsection{main.cpp}
\lstinputlisting[style = cppstyle]{./../main.cpp}
\clearpage
\subsection{Test.hpp}
\lstinputlisting[style = cppstyle]{./../Test.hpp}

% Literaturverzeichnis
% --------------------
%\begin{thebibliography}{99}
%\bibitem{Pomberger} Pomberger G., Blaschek G. : \textit{Software Engineering: Prototyping und objektorientert Software-Entwicklung}. Hanser, 1996
%
%\end{thebibliography}


% Ende des Dokuments
% ------------------
\end{document}
