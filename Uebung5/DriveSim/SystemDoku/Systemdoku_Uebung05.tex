% Systemdokumentation OOP3
% Unterlage für Studenten als Leitfaden für die Erstellung einer SystemDoku
% 17. Oktober 2022
% ---------------------------------------------------------------------------



% Dokumentklasse
% --------------
\documentclass[12pt,naustrian,a4widepaper]{scrartcl}   
% article style
%   - 11pt Schriftgroesse
%   - new austrian (neue Rechtschreibung)
%   - Papierformat A4
%   - pdf-hyperlinks


% Packages
% --------
\usepackage[utf8]{inputenc}  % fuer Umlaute, Ü
\usepackage[T1]{fontenc}
\usepackage{a4wide}
\usepackage{times}      % Times Schriften (zusammen mit fontencoding, s.o.)

\usepackage{babel}
\usepackage{graphicx}	  % für das Einbinden von Grafiken
\usepackage{color}      % für färbigen Text
\usepackage{framed}     % für (Text-) Rahmen
\usepackage{fancyhdr}   % für Kopf- und Fusszeilen
\usepackage{listings}   % für den Sourcecode
\usepackage{pdfpages}
\usepackage{rotating}

\pagestyle{fancy}       % Kopf- / Fusszeile aktivieren

\definecolor{failred}{rgb}{0.7, 0.0, 0.0}
\definecolor{okgreen}{rgb}{0.0, 0.5, 0.0}
\definecolor{gray}{rgb}{0.0, 0.5, 0.0}

\lstdefinelanguage{TestOutput}{
    morekeywords={},
    morecomment=[l]{//},
    morestring=[b]",
    sensitive=false,
}

\lstdefinestyle{teststyle}{
    language=TestOutput,
    basicstyle=\ttfamily\footnotesize,
    keywordstyle=\color{black},
    commentstyle=\color{gray},
    stringstyle=\color{black},
    showstringspaces=false,
    breaklines=true,
    frame=single,
    numbers=left,
    numberstyle=\tiny\color{gray},
    postbreak=\mbox{\textcolor{red}{$\hookrightarrow$}\space},
    literate={OK}{{\textcolor{okgreen}{OK}}}2
             {Fail}{{\textcolor{failred}{Fail}}}4,
}

\lstset{
	language={C++},
	basicstyle=\tiny\ttfamily,
	keywordstyle=\color{blue},%\bfseries,
	commentstyle=\color{green},
	frame=single,
	linewidth=16cm,
	breaklines=false,
	tabsize=3,
	numbers=left, numberstyle=\tiny, stepnumber=1, numbersep=5pt
}


% Seitenspiegel
% -------------

\typearea{8}	% Festlegung des Seitenspiegels gem. Koma. 4..groß, 9..klein


% Kopfzeile
% ---------
\lhead{{\footnotesize{s. Offenberger, S. Vogelhuber}}}   %  (links)
\chead{{\footnotesize{Systemdokumentation - DriveSim}}} %  (mitte)
\rhead{{\footnotesize{Seite \thepage}}}      %  (rechts)

% Fusszeile
% ---------
\lfoot{}  % links
\cfoot{}  % mitte 
\rfoot{}  % rechts

% Absatzformatierung
% ------------------
\setlength{\parindent}{0cm}   % Einrückung der 1. Zeile jedes Absatzes
\setlength{\parskip}{10pt}    % Abstand zwischen den Absätzen


% Package für Hyperlinks (mit pdf-Optionen)
% -----------------------------------------
\usepackage[
urlcolor=blue,		% blaue weblinks
linkcolor=black,	% interne Links sind schwarz
colorlinks=true,        % links werden eingefärbt
pdfstartview=FitH,      % PDF-Anzeige: Fensterbreite
pdfborder={0 0 0},      % keine Umrandung um links
pdftitle   ={Systemdokumentation},% Referenzen in der Pdf-Datei
pdfauthor  ={M. Mustermann, S. Sorglos},
pdfsubject ={Systemdoku},
pdfcreator ={Der Creator},
pdfproducer={Der Producer},
pdfkeywords={Dokumentation, Systemdokumentation}
]{hyperref}


% Beginn des Dokumentes
% ---------------------
\begin{document}

\selectlanguage{naustrian}   % oder "american" für engl. Texte

% Titelblatt
% ----------
\title {\vspace{1cm}
       \includegraphics[width=8cm]{./Images/FhOOeLogoOkt2009_HSD_Rot_pastell}\\
       \vspace{2cm}
       {\textbf{Systemdokumentation\\Projekt DriveSim}}\\
       \vspace{5mm}
       {\small{Version 1.0}}\\
       \vspace{5mm}
}

\author{\small{S. Offenberger, S. Vogelhuber}}
\date  {\small{Hagenberg, \today}}
\maketitle

%\begin{abstract}
%Dieses Dokument zeigt den prinzipiellen Aufbau einer Systemdokumentation für Software-Projekte. Die einzelnen Kapitel sind mit Kommentaren versehen, welche die Struktur und den Inhalt erläutern. 
%\end{abstract}

\clearpage

% Inhalts-, Tabellen- und Bildverzeichnis (werden generiert)
% ----------------------------------------------------------
\tableofcontents
% \listoftables
% \listoffigures
\clearpage



\section{Organisatorisches}

\subsection{Team}
\begin{itemize}
	\item Simon Offenberger, Matr.-Nr.: S2410306027, E-Mail: Simon.Offenberger@fh-hagenberg.at
	\item Simon Vogelhuber, Matr.-Nr.: S2410306014, E-Mail: S2410306014@fh-hagenberg.at	
\end{itemize}

\subsection{Aufteilung der Verantwortlichkeitsbereiche}
\begin{itemize}
	\item Simon Offenberger
		\begin{itemize}
			\item Design Klassendiagramm
			\item Implementierung des Testtreibers
			\item Dokumentation
			\item Implementierung und Test der Klassen: 
			\begin{itemize}
				\item Object,
				\item Vehicle,
				\item Car,
				\item Meter,
				\item IDisplay,
				\item Tachometer,
				\item Odometer,
				\item RPM Sensor
			\end{itemize}
		\end{itemize}
	\item Simon Vogelhuber
	\begin{itemize}
			\item Design Klassendiagramm
			\item Implementierung des Testtreibers
			\item Dokumentation
			\item Implementierung und Test der Klassen: 
			\begin{itemize}
				\item Object,
				\item Vehicle,
				\item Car,
				\item Meter,
				\item IDisplay,
				\item Tachometer,
				\item Odometer,
				\item RPM Sensor
			\end{itemize}
		\end{itemize}	
\end{itemize}

\subsection{Aufwand}
	
	\begin{itemize}
		\item Simon Offenberger: geschätzt 10 Ph / tatsächlich x Ph
		\item Simon Vogelhuber:  geschätzt 10 Ph / tatsächlich 6 Ph
	\end{itemize}

\clearpage
\section{Anforderungsdefinition (Systemspezifikation)}
In diesem Abschnitt werden die funktionalen Anforderungen an das System beschrieben.

\textbf{Funktionen des Vehicle}
\begin{itemize}
	\item Abstrakte Basisklasse für alle Fahrzeugtypen
	\item Anfragen der Sensordaten (Drehzahl der Reifen)
\end{itemize}

\textbf{Funktionen des Sensors}
\begin{itemize}
	\item Erzeugen der Sensordaten (Drehzahl der Reifen) durch lesen einer Textdatei
	\item Weitergabe der Sensordaten an die verbauten Anzeigen
\end{itemize}

\textbf{Funktionen des Display}
\begin{itemize}
	\item Interface für alle Anzeigetypen für die Verwendung eines Displays innerhalb eines Vehicles
\end{itemize}

\textbf{Funktionen des Meter}
\begin{itemize}
	\item Abstrakte Basisklasse für alle Anzeigetypen
	\item Anbfragen der anzuzeigenden Werte vom Vehicle
	\item Darstellung der Werte
\end{itemize}

\textbf{Funktionen des Odometer}
\begin{itemize}
	\item Berechnung der gefahrenen Strecke anhand der Raddrehzahl
\end{itemize}

\textbf{Funktionen des Tachometer}
\begin{itemize}
	\item Messung der Drehzahl des Motors
\end{itemize}

\clearpage
\section{Systementwurf}

\begin{sidewaysfigure}
    \centering
    {\large \textbf{Klassendiagramm}\\[1em]}
    \includegraphics[width=\textheight]{./Images/Klassendiagramm.png}
\end{sidewaysfigure}

\subsection{Designentscheidungen}
Im Klassendiagramm wurd als Pattern das \textit{Observer Pattern} verwendet, da in diesem fall ein Fahzeug
mehrere Anzeigen (Meter) haben kann, welche auf die Daten des Fahrzeugs reagieren müssen. Das Observer 
Pattern ermöglicht es, dass die Anzeigen sich beim Fahrzeug registrieren können und bei einer Änderung 
der Sensordaten automatisch benachrichtigt werden.

Die Displays können vom Fahrzeug die aufbereiteten Daten (z.B. Geschwindigkeit, Drehzahl) abfragen,
indem sie die entsprechenden Methoden des Fahrzeugs aufrufen.

Die Abstakte \textit{Meter} dient dazu, das WindowDisplay Interface zu implementieren und die gemeinsamen
Funktionen der verschiedenen Anzeigetypen (Tachometer, Odometer, RPM Sensor) zu bündeln.

\color{black}

\section{Dokumentation der Komponenten (Klassen)}
Die HTML-Startdatei befindet sich im Verzeichnis \href{run:./../doxy/html/index.html}{./../doxy/html/index.html}

\clearpage
\section{Testprotokollierung}
\lstinputlisting[style=teststyle]{../output.txt}

\clearpage
\section{Quellcode}

\subsection{Object.h}
\lstinputlisting{../Object.h}
\clearpage

\subsection{Vehicle.hpp}
\lstinputlisting{../Vehicle.hpp}
\clearpage

\subsection{Vehicle.cpp}
\lstinputlisting{../Vehicle.cpp}
\clearpage

\subsection{Car.hpp}
\lstinputlisting{../Car.hpp}
\clearpage

\subsection{Car.cpp}
\lstinputlisting{../Car.cpp}
\clearpage

\subsection{IDisplay.hpp}
\lstinputlisting{../IDisplay.hpp}
\clearpage

\subsection{Meter.hpp}
\lstinputlisting{../Meter.hpp}
\clearpage

\subsection{Meter.cpp}
\lstinputlisting{../Meter.cpp}
\clearpage

\subsection{Tachometer.hpp}
\lstinputlisting{../Tachometer.hpp}
\clearpage

\subsection{Tachometer.cpp}
\lstinputlisting{../Tachometer.cpp}
\clearpage

\subsection{Odometer.hpp}
\lstinputlisting{../Odometer.hpp}
\clearpage

\subsection{Odometer.cpp}
\lstinputlisting{../Odometer.cpp}
\clearpage

\subsection{RPM\_Sensor.hpp}
\lstinputlisting{../RPM_Sensor.hpp}
\clearpage

\subsection{RPM\_Sensor.cpp}
\lstinputlisting{../RPM_Sensor.cpp}
\clearpage

\subsection{main.cpp}
\lstinputlisting{../main.cpp}
\clearpage

\subsection{Test.hpp}
\lstinputlisting{../Test.hpp}
\clearpage

% Literaturverzeichnis
% --------------------
%\begin{thebibliography}{99}
%\bibitem{Pomberger} Pomberger G., Blaschek G. : \textit{Software Engineering: Prototyping und objektorientert Software-Entwicklung}. Hanser, 1996
%
%\end{thebibliography}


% Ende des Dokuments
% ------------------
\end{document}
