% Systemdokumentation OOP3
% Unterlage für Studenten als Leitfaden für die Erstellung einer SystemDoku
% 17. Oktober 2022
% ---------------------------------------------------------------------------



% Dokumentklasse
% --------------
\documentclass[12pt,naustrian,a4widepaper]{scrartcl}   
% article style
%   - 11pt Schriftgroesse
%   - new austrian (neue Rechtschreibung)
%   - Papierformat A4
%   - pdf-hyperlinks


% Packages
% --------
\usepackage[utf8]{inputenc}  % fuer Umlaute, Ü
\usepackage[T1]{fontenc}
\usepackage{a4wide}
\usepackage{times}      % Times Schriften (zusammen mit fontencoding, s.o.)

\usepackage{babel}
\usepackage{graphicx}	  % für das Einbinden von Grafiken
\usepackage{color}      % für färbigen Text
\usepackage{framed}     % für (Text-) Rahmen
\usepackage{fancyhdr}   % für Kopf- und Fusszeilen
\usepackage{listings}   % für den Sourcecode
\usepackage{pdfpages}

\pagestyle{fancy}       % Kopf- / Fusszeile aktivieren

\definecolor{failred}{rgb}{0.7, 0.0, 0.0}
\definecolor{okgreen}{rgb}{0.0, 0.5, 0.0}
\definecolor{gray}{rgb}{0.0, 0.5, 0.0}

\lstdefinelanguage{TestOutput}{
    morekeywords={},
    morecomment=[l]{//},
    morestring=[b]",
    sensitive=false,
}

\lstdefinestyle{teststyle}{
    language=TestOutput,
    basicstyle=\ttfamily\footnotesize,
    keywordstyle=\color{black},
    commentstyle=\color{gray},
    stringstyle=\color{black},
    showstringspaces=false,
    breaklines=true,
    frame=single,
    numbers=left,
    numberstyle=\tiny\color{gray},
    postbreak=\mbox{\textcolor{red}{$\hookrightarrow$}\space},
    literate={OK}{{\textcolor{okgreen}{OK}}}2
             {Fail}{{\textcolor{failred}{Fail}}}4,
}

\lstset{
	language={C++},
	basicstyle=\tiny\ttfamily,
	keywordstyle=\color{blue},%\bfseries,
	commentstyle=\color{green},
	frame=single,
	linewidth=16cm,
	breaklines=false,
	tabsize=3,
	numbers=left, numberstyle=\tiny, stepnumber=1, numbersep=5pt
}


% Seitenspiegel
% -------------

\typearea{8}	% Festlegung des Seitenspiegels gem. Koma. 4..groß, 9..klein


% Kopfzeile
% ---------
\lhead{{\footnotesize{s. Offenberger, S. Vogelhuber}}}   %  (links)
\chead{{\footnotesize{Systemdokumentation - Filesystem}}} %  (mitte)
\rhead{{\footnotesize{Seite \thepage}}}      %  (rechts)

% Fusszeile
% ---------
\lfoot{}  % links
\cfoot{}  % mitte 
\rfoot{}  % rechts

% Absatzformatierung
% ------------------
\setlength{\parindent}{0cm}   % Einrückung der 1. Zeile jedes Absatzes
\setlength{\parskip}{10pt}    % Abstand zwischen den Absätzen


% Package für Hyperlinks (mit pdf-Optionen)
% -----------------------------------------
\usepackage[
urlcolor=blue,		% blaue weblinks
linkcolor=black,	% interne Links sind schwarz
colorlinks=true,        % links werden eingefärbt
pdfstartview=FitH,      % PDF-Anzeige: Fensterbreite
pdfborder={0 0 0},      % keine Umrandung um links
pdftitle   ={Systemdokumentation},% Referenzen in der Pdf-Datei
pdfauthor  ={M. Mustermann, S. Sorglos},
pdfsubject ={Systemdoku},
pdfcreator ={Der Creator},
pdfproducer={Der Producer},
pdfkeywords={Dokumentation, Systemdokumentation}
]{hyperref}


% Beginn des Dokumentes
% ---------------------
\begin{document}

\includepdf[pages= 1-2]{Uebung06_Deckblatt.pdf}


\selectlanguage{naustrian}   % oder "american" für engl. Texte

% Titelblatt
% ----------
\title {\vspace{1cm}
       \includegraphics[width=8cm]{./Images/FhOOeLogoOkt2009_HSD_Rot_pastell}\\
       \vspace{2cm}
       {\textbf{Systemdokumentation\\Projekt Filesystem}}\\
       \vspace{5mm}
       {\small{Version 1.0}}\\
       \vspace{5mm}
}

\author{\small{S. Offenberger, S. Vogelhuber}}
\date  {\small{Hagenberg, \today}}
\maketitle

%\begin{abstract}
%Dieses Dokument zeigt den prinzipiellen Aufbau einer Systemdokumentation für Software-Projekte. Die einzelnen Kapitel sind mit Kommentaren versehen, welche die Struktur und den Inhalt erläutern. 
%\end{abstract}

\clearpage

% Inhalts-, Tabellen- und Bildverzeichnis (werden generiert)
% ----------------------------------------------------------
\tableofcontents
% \listoftables
% \listoffigures
\clearpage



\section{Organisatorisches}

\subsection{Team}
\begin{itemize}
	\item Simon Offenberger, Matr.-Nr.: S2410306027, E-Mail: Simon.Offenberger@fh-hagenberg.at
	\item Simon Vogelhuber, Matr.-Nr.: S2410306014, E-Mail: Simon.Vogelhuber@fh-hagenberg.at	
\end{itemize}

\subsection{Aufteilung der Verantwortlichkeitsbereiche}
\begin{itemize}
	\item Simon Offenberger
		\begin{itemize}
			\item Design Klassendiagramm
			\item Implementierung und Test der Klassen: 
			\begin{itemize}
				\item IVisitor,
				\item FilterVisitor, 
				\item FilterFileVisitor,
				\item FilterLinkVisitor,
				\item DumpVisitor und
				\item FSObjectFactory
			\end{itemize}
			\item Implementierung des Testtreibers
			\item Dokumentation
		\end{itemize}
	\item Simon Vogelhuber
		\begin{itemize}
			\item Design Klassendiagramm
			\item Implementierung und Komponententest der Klassen: 
			\begin{itemize}
				\item FSObject
				\item File, 
				\item iFolder, 
				\item Folder und  
				\item Link 
			\end{itemize}
			\item Implementierung des Testtreibers
			\item Dokumentation
		\end{itemize}	
\end{itemize}

\subsection{Aufwand}
	
	\begin{itemize}
		\item Simon Offenberger: geschätzt 7 Ph / tatsächlich 9 Ph
		\item Simon Vogelhuber:  geschätzt 8 Ph / tatsächlich 7 Ph
	\end{itemize}

\clearpage
\section{Anforderungsdefinition (Systemspezifikation)}

Das zu entwickelnde System dient der Simulation eines einfachen Dateisystems für ein
eingebettetes System. Ziel ist es, die Struktur und das Verhalten eines hierarchischen
Dateisystems softwaretechnisch abzubilden und durch geeignete Entwurfsmuster
(Composite, Factory, Visitor) erweiterbar und wartbar zu gestalten.  
Die Anforderungen ergeben sich aus der gegebenen Systemspezifikation der Übung
und beschreiben die grundlegenden funktionalen Eigenschaften des Systems
sowie alle wesentlichen Rahmenbedingungen für das Verständnis.

\subsection{Systemüberblick}
Das System verwaltet drei Arten von Dateisystemelementen:
\begin{itemize}
    \item \textbf{Dateien}
    \item \textbf{Ordner}
    \item \textbf{Verweise} (Referenzen auf Dateien, Ordner oder weitere Verweise)
\end{itemize}

Diese Elemente bilden gemeinsam eine hierarchische Struktur, in der Ordner beliebige
Kombinationen dieser Elemente enthalten können. Jedes Element besitzt einen Namen,
der nachträglich veränderbar ist.

\subsection{Funktionale Anforderungen}

\subsubsection{Dateien}
Eine Datei verfügt über folgende unveränderliche Eigenschaften, die bei ihrer Erzeugung
festgelegt werden:
\begin{itemize}
    \item Blockgröße auf dem Speichermedium (Bytes)
    \item Anzahl reservierter Blöcke
\end{itemize}
Zusätzlich wird die aktuelle Dateigröße in Bytes verwaltet.  
Das Schreiben in eine Datei erfolgt über:
\begin{itemize}
    \item \texttt{Write(size\_t const bytes)}
\end{itemize}
Die Datei darf niemals größer werden als der durch die reservierten Blöcke bereitgestellte Speicher.

\subsubsection{Ordner}
Ein Ordner kann beliebig viele Dateien, Verweise und weitere Ordner enthalten.
Er bildet die Grundlage des hierarchischen Dateisystems.

\subsubsection{Verweise}
Ein Verweis referenziert exakt ein Zielobjekt (Datei, Ordner oder weiteren Verweis).
Der Name des Verweises kann verändert werden, zusätzlich muss der Name des Zielobjekts
im Rahmen der Filterausgabe ausgegeben werden.

\subsection{Erzeugung der Elemente}
Für die Erstellung aller Dateisystemelemente ist eine einfache \textbf{Fabrik} zu implementieren.
Diese kapselt die Instanziierungslogik und stellt sicher, dass die Objekterzeugung einheitlich erfolgt.

\subsection{Besucher (Visitor) Anforderungen}

\subsubsection{Visitor: Dump}
\begin{itemize}
    \item Gibt die gesamte Dateisystemhierarchie aus.
    \item Ausgabe sowohl auf der Standardausgabe als auch in einer Datei.
    \item Muss Dateien, Ordner und Verweise in strukturierter Form darstellen.
\end{itemize}

\subsubsection{Visitor: FilterFiles}
\begin{itemize}
    \item Filtert Dateien anhand eines minimalen und maximalen Größenschwellwerts.
    \item Filtert optional alle Verweise.
    \item Ausgabe aller gefilterten Dateien mit ihrem vollständigen Pfad.
    \item Bei Verweisen muss zusätzlich der Name des referenzierten Zielobjekts ausgegeben werden.
\end{itemize}

\subsection{Testanforderungen}
Ein Testtreiber muss:
\begin{itemize}
    \item ein mehrstufiges hierarchisches Dateisystem erzeugen,
    \item alle implementierten Besucher umfassend testen,
    \item die Funktionsweise demonstrieren und Testausgaben bereitstellen.
\end{itemize}



\clearpage
\section{Systementwurf}

\subsection{Klassendiagramm}
\includegraphics[width=14cm]{./FileSystem.png}
\newpage

\subsection{Designentscheidungen}


\color{black}

\section{Dokumentation der Komponenten (Klassen)}
Die HTML-Startdatei befindet sich im Verzeichnis \href{run:./../doxy/html/index.html}{./../doxy/html/index.html}

\clearpage
\section{Testprotokollierung}
\lstinputlisting[style=teststyle]{../TestOutput.txt}

\clearpage
\section{Quellcode}

\subsection{Object.hpp}
\lstinputlisting{../Object.h}
\clearpage
\subsection{FSObjectFactory.hpp}
\lstinputlisting{./../FSObjectFactory.hpp}
\clearpage
\subsection{FSObjectFactory.cpp}
\lstinputlisting{./../FSObjectFactory.cpp}

\clearpage
\subsection{Filesystem.hpp}
\lstinputlisting{./../Filesystem.hpp}

\clearpage
\subsection{Filesystem.cpp}
\lstinputlisting{./../Filesystem.cpp}

\clearpage
\subsection{FSObject.hpp}
\lstinputlisting{./../FSObject.hpp}

\clearpage
\subsection{FSObject.cpp}
\lstinputlisting{./../FSObject.cpp}
\clearpage

\subsection{File.hpp}
\lstinputlisting{./../File.hpp}
\clearpage

\subsection{File.cpp}
\lstinputlisting{./../File.cpp}
\clearpage

\subsection{IFolder.hpp}
\lstinputlisting{./../IFolder.hpp}
\clearpage

\subsection{Folder.hpp}
\lstinputlisting{./../Folder.hpp}
\clearpage

\subsection{Folder.cpp}
\lstinputlisting{./../Folder.cpp}
\clearpage

\subsection{Link.hpp}
\lstinputlisting{./../Link.hpp}
\clearpage

\subsection{Link.cpp}
\lstinputlisting{./../Link.cpp}
\clearpage

\subsection{IVisitor.hpp}
\lstinputlisting{./../IVisitor.hpp}
\clearpage

\subsection{FilterVisitor.hpp}
\lstinputlisting{./../FilterVisitor.hpp}
\clearpage

\subsection{FilterVisitor.cpp}
\lstinputlisting{./../FilterVisitor.cpp}
\clearpage

\subsection{FilterFileVisitor.hpp}
\lstinputlisting{./../FilterFileVisitor.hpp}
\clearpage

\subsection{FilterFileVisitor.cpp}
\lstinputlisting{./../FilterFileVisitor.cpp}
\clearpage

\subsection{FilterLinkVisitor.hpp}
\lstinputlisting{./../FilterLinkVisitor.hpp}
\clearpage

\subsection{FilterLinkVisitor.cpp}
\lstinputlisting{./../FilterLinkVisitor.cpp}
\clearpage

\subsection{DumpVisitor.hpp}
\lstinputlisting{./../DumpVisitor.hpp}
\clearpage

\subsection{DumpVisitor.cpp}
\lstinputlisting{./../DumpVisitor.cpp}
\clearpage

\subsection{main.cpp}
\lstinputlisting{./../main.cpp}
\clearpage

\subsection{Test.hpp}
\lstinputlisting{./../Test.hpp}
	

% Literaturverzeichnis
% --------------------
%\begin{thebibliography}{99}
%\bibitem{Pomberger} Pomberger G., Blaschek G. : \textit{Software Engineering: Prototyping und objektorientert Software-Entwicklung}. Hanser, 1996
%
%\end{thebibliography}


% Ende des Dokuments
% ------------------
\end{document}
